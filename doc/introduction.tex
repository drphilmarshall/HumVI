%=================================================

% Introduction to HumVI paper

\documentclass[letterpaper, 11pt]{article}

%=================================================

\usepackage{amsmath,amssymb}
\usepackage[round]{natbib}
\usepackage{aas_macros}
\usepackage{xspace}
\usepackage{hyperref}

%=================================================

\def\HST{HST\xspace}

%=================================================

\title{Discovery Sensitivity in Human Viewable Images}
\author{Cato Sandford\thanks{\texttt{cato@nyu.edu}},
        Phil Marshall,
    and David W.\ Hogg}
\date{\today}

%%-------------------------------------------------

\begin{document}

\maketitle

%%=================================================
%% Introduction
%%=================================================

\section{Introduction}

In astronomy, as in many other fields of science, discoveries can be made by
simply looking at images. New and unusual astronomical objects, such as
galaxies, nebulae, star clusters, supernovae and so on, appear readily to
experienced viewers as they inspect new images \citep[a recent example is the amateur discovery of a quadruple star system, reported in][]{PH12}; our understanding of  the structure of, for example, galaxies can
be improved by studying their morphology, by eye\footnote{{\bf Phil:} indeed, is it not true that there is sometimes little alternative if one wants to draw meaningful conclusions about galaxy structure? Perhaps it is beyond the scope here, but it may be interesting to discuss what can be done by computers and what cannot -- the question isn't really addressed head-on even in your theory paper, yet it provides a compelling motivation for this work.} \citep[e.g.][]{Lah95, Lah++95, For++11, But11}\footnote{{\bf Phil}: We really want evidence of \emph{visual} inspection being useful, hence these references. Possibly too many.}.
Image viewing is a form of data exploration, an important step before making
quantitative measurements. But both exploration and measurement involve data
modeling, or inference: the viewer interprets the image in the context of a
model for that image that they hold.  By investigating and modeling this
viewing process, we can hope to increase the efficiency of such discoveries,
increasing the rate at which true detections are made, and reducing the
incidence of false detections. 

One way to increase the frequency with which discoveries are made  could be to
employ a larger number of experienced image viewers to look at images.  This
approach is being taken by ``citizen science'' projects such as Galaxy Zoo
(citation required\footnote{Cato, Phil: Add more zoo citations. In paper X they
found Y, in paper Z they found...}). In the Galaxy Zoo system, color composite
images of galaxies are presented to large numbers of viewers, who carry out
visual morphological studies guided by a short questionnaire. The viewers have
a wide range of experience with astronomical images, and the majority of
viewers come to the site not having viewed many astronomical images before.
The model for astrophysical objects that this group has is therefore
data-driven: their ability to spot something new depends partly on their
pre-conceived notions of what galaxies look like, but to a greater extent it
will depend on the images they have seen in the system before. Given an
inspection Zoo and a user base like this,  how should we prepare and present
the color images of galaxies, to increase the probability of an interesting
new feature being detected? This is the question we seek to answer in this
paper, taking as our archetypal interesting features the faint arcs caused by
gravitational lensing. 

Gravitational lenses enable a wide range of science projects, providing a
means to measure the mass distributions of  galaxies, groups and clusters,
independent of their luminosity or dynamical state, and giving us a rare
magnified view of the distant universe. At present almost all these projects
are limited by the small samples of lenses known, but the wide field surveys
coming online have the potential to change that. A number of optical and near
infra-red imaging surveys are planned for the next decade that together have
been predicted to contain over 10,000 new gravitational lenses, an increase in
sample size relative to the present day of around two orders of magnitude
(see, for example, \citet{O+M10}; \citet{Paw++12}\footnote{The lensing discoveries of \citet{Paw++12} arose from the dedicated visual inspection of \HST images by two astronomers. They note that for the next generation of telescope missions, the time-demands of their method will be (even more) impractical. But their work lends credence to the claim that visual inspection may yet make valuable contributions to source-discovery.}; \citet[chapter 12]{LSST09}; \citet[chapter 1, page 8]{ERB10}). Most of these
surveys will be carried out on ground-based telescopes, and will be both
multi-filter, in many cases multi-epoch, and synoptic, in order to meet a wide
range of different science requirements. The thousands of square degrees of
sky imaged, and the billions of objects catalogued, will provide a huge mine
of data to be searched for rare objects like lenses. We expect image viewing
to play a role in this search process, most likely in the form of quality
control inspection of the outputs of various automated detection algorithms.

For an image to enable discovery, it needs to be {\it informative}: that is,
it must have high quality, so that interesting features are visible to the
viewer, and it must not be confusing, so that interesting features are not
mistaken for artifacts and ignored. The quality of an astronomical image is
only partly determined by the observing conditions, telescope, and camera: the
image processing carried out in software is also important. We have additional
information about the image that we can use to improve both its resolution and
depth. The stars provide images of the imaging system point spread function
(PSF), while our understanding of the detector and the sky background provide
a model for the noise in the image: we can attempt to use both of these in
reconstructing a higher quality image. There is a significant body of
literature on this image restoration process in astronomy \citep[e.g.][]{Ric72, N+N82, S+B84, P+P93, MCS98}. We might
expect algorithms like this to be important for synoptic surveys, whose
resolution and depth varies from image to image, and from filter to filter in
a partiular field of view. Combining raw images into a color composite will
produce colored artifacts due to the mis-matched resolution in the red, green
and blue channels; while the resulting composite will have different (mean)
resolution than other elsewhere on the sky, leading to an inhomogeneity that
will hinder the development of the viewers' internal feature model as they
have to allow for the variations in image quality. However, deconvolution is
notorious for producing image artifacts if not sufficiently regularized
\citep[see, for example, comments at the end of section~1 of][]{MCS98}: such artifacts could
create more problems than the deconvolution solves, reducing the probability
of a discovery being made.  An experimental test of the sensitivity with which
an image set enables discovery is required.

In this paper we investigate a simple model-based deconvolution scheme for
resolution matching, combined with a standardised image scaling and stretch,
producing homogenized sets of color composites for inspection and then testing
their sensitivity for lensed feature detection by viewers in the Galaxy Zoo.
Using a set of $N$ realistic simulated test images for the XXX survey 
containing faint lensed features, we define a set of metrics that quantify
discovery sensitivity based on a test group of viewers responses, and 
ask the following questions:

\begin{itemize}

\item Does simple, ``light'' deconvolution designed for making
resolution-matched composite images introduce significant colored artifacts?

\item What target matched resolution PSF width should be used in order to
maximise discovery sensitivity in the color composite images? How does this
relate to the input images' resolution?

\item What algorithm for choosing the color composite images' stretch and
scaling should be used, to to maximise discovery sensitivity? 

\item Is there significant scatter in viewers preferences regarding image
stretching and scaling, such that viewer control over these parameters is
justified?

\end{itemize}

While we focus on lensed features as our discovery targets, and the XXX
survey, we hope that our results will be of interest to the astronomical
community in general, and in particular to anyone who wants to see what they
have found in the object catalog database of a large synoptic imaging survey.

In Section... we...

\newpage
\bibliographystyle{plainnat}
\bibliography{humvi}

\end{document}
